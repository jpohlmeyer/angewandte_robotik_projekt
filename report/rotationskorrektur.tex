\section{Rotationskorrektur}

zunächst nur mit odometriedaten korrigieren -> klappt nicht: globalen offset mit speichern und erhöhen weil in jedem schritt nur relativer offset berechnet wird

dann finden des lokalen maximums der korrelation mithilfe der odometriedaten als ausgangspunkt für die suche da korrelation nur eindeutig, wenn keine Translation

erster ansatz: gleichzeitig nach links und nach rechts vom odometriepunkt suchen, bis ein wert gefunden wurde, der mindestens 50\% des maximalen korrelationswert hat. später hochgesetzt auf 2/3 des maximalen korrelationswert hochgesetzt -> idee: kein lokales maximum im rauschen finden sondern klares maximum finden

funktioniert manchmal nicht ganz so gut, manchmal maxima nicht gefunden, da dies zu klein ist (spaeter durch rauschen vermutlich noch schlimmer)..

neuer ansatz: in 15 grad um den odometriepunkt den maximalen punkt in der korrelation suchen -> besser

entsprechend dem lokal maximalen bin der korrelation den scan rotieren und die rotation auf den globalen offset aufaddieren und den scan in die karte eintragen