\section{X/Y Histogramme (Jan)}

Nachdem der Scan auf die Hauptachsen ausgerichtet wurde kann ein X- und Y-Histogramm erstellt werden. Diese Histogramme stellen eine Statistik über die Anzahl der Punkte in X- bzw. Y-Richtung auf.
Dadurch, dass der Scan auf die Hauptachsen ausgerichtet wurde, fallen die X- bzw. Y-Koordinaten der Punkte auf einer Wand immer auf denselben X- bzw. Y-Wert. Es bildet sich nun für jede Wand ein Maximum in einem der beiden Histogramme. Wichtig, damit diese Methode funktioniert ist, dass die Wände gerade sind und sich an den Ecken 90\degree Winkel befinden. Falls eine Wand nicht parallel zu einer der Hauptachsen liegt, verteilen sich die Punkte der Wand und bilden kein Maximum.


auf dem achsenalignten scan werden die scan punkte in ein x und ein y histogram eingetragen

die größe der histogramme wird dabei an der maximalen laserdistanz ausgerichtet damit alle werte eingetragen werden können