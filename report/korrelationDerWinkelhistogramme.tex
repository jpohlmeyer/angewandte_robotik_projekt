\subsection{Korrelation der Winkelhistogramme (Janneke)}

Im letzten Kapitel haben wir gezeigt, wie wir die Winkelhistogramme berechnet haben. Nun müssen wir mithilfe dieser die Rotation von zwei aufeinander folgenden Scans zueinander berechnen. Dazu berechnet man die Korrelation der Scans mit folgender Formel bei der $s_1$ den alten Scan, $s_2$ den aktuellen Scan und $k(j)$ die Korrelation mit Verschiebung j angibt:$$k(j) = \sum_{i = 1}^{n} s_1(i)*s_2(i+j)$$
Wir verschiebt den neuen Scan gegen den alten und summiert das Produkt der beiden Histogrammwerte eines Bins auf. Durch die Multiplikation ist die Korrelation hoch, wenn Peaks in den Histogrammen übereinander liegen, da in diesem Fall zwei relativ hohe Werte multipliziert werden. Sie ist dagegen gering, wenn die Peaks nicht übereinstimmen, da in diesem Fall der hohe Wert eines Peaks in dem einen Histogramm mit einem geringen Wert (evtl. sogar Null) aus dem anderen Histogramm multipliziert wird. An der Korrelation kann man die relative Verschiebung der beiden betrachteten Scans erkennen. Die Korrelation ist wie die Winkelhistogramme auch diskret, aber sie geht nicht von -180 bis +180 Grad, sondern von einer Verschiebung 0 bis 2$\pi$. In unserer Ausgabe haben wir die Korrelationswerte mit dem höchsten Wert und der Fenstergröße skaliert, damit immer alle Korrelationspunkte sichtbar sind.

%TODO:
%\begin{figure}
%	\centering
%	\includegraphics[width=11cm]{winkelhistogrammeMitKorrelation}
%	\caption{Winkelhistogramme und deren Korrelation}
%	\label{fig:Korrelation}
%\end{figure}

Abbildung %~\ref{fig:Korrelation}% 
zeigt unser Ergebnis, wobei das aktuelle Winkelhistogramm mit grünen Punkten, das Winkelhistogramm vom vorherigen Scan mit blauen Punkten und die Korrelation der beiden Histogramme mit roten Punkten dargestellt ist.