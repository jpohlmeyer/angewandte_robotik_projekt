\subsection{X- und Y-Histogramme (Jan)}

Nachdem der Scan auf die Hauptachsen ausgerichtet wurde, kann ein X- und Y-Histogramm erstellt werden. Diese Histogramme stellen eine Statistik über die Anzahl der Punkte in X- bzw. Y-Richtung auf.
Dadurch, dass der Scan auf die Hauptachsen ausgerichtet wurde, fallen die X- bzw. Y-Koordinaten der Punkte auf einer Wand immer auf denselben X- bzw. Y-Wert. Es bildet sich nun für jede Wand ein Maximum in einem der beiden Histogramme. Wichtig, damit diese Methode funktioniert ist, dass die Wände gerade sind und sich an den Ecken 90 Grad Winkel befinden. Falls eine Wand nicht parallel zu einer der Hauptachsen liegt, verteilen sich die Punkte der Wand entlang der Achsen und bilden kein Maximum. Die Größe der im Histogramm abbildbaren Werte wird anhand der maximal messbaren Distanz in dem vorliegenden Raum angepasst. Diese muss sowohl in positiver Achsenrichtung, als auch in negativer Richtung aufgetragen werden können. In der Mitte der Histogramme befindet sich der Nullpunkt. In Abbildung~\ref{fig:xyhistogram} kann man im X-Histogramm sehen, dass sich sowohl links als auch rechts vom Roboter eine Wand befindet. Im Y-Histogramm lassen sich zwei Wände in unterschiedlicher Distanz vor dem Roboter erkennen.