\section{Parameteroptimierung}

Da sich nun keine implementatorischen Fehler im Programm befanden konnten wir dazu übergehen die unterschiedlichen Parameter so anzupassen, dass die Karte eines durch Pappkartons errichteten Bereiches im Labor optimal aufgenommen wird.

\subsection{Optimierung des COUNT-Parameters (Jan)}

Der COUNT-Parameter legt fest bei welchen Schleifendurchläufen zusätzlich zur Hindernisvermeidung auch ein Karten-Update durchgeführt wird. Dies ist zum einen notwendig, damit der Roboter mit den Scans hinterherkommt und zum anderen sinnvoll, da der Roboter dann bereits eine bedeutendere Bewegung durchgeführt hat, sodass in der Differenz der Scans die Bewegungsänderung signifikanter als das Rauschen ist. Wir haben zusätzlich einen kurzen Sleep von 100ms in die Hauptschleife eingefügt, damit der Roboter mit den Scans nicht in Verzug gerät, aber trotzdem zeitnah auf auftauchende Hindernisse reagieren kann.
Wir haben den COUNT-Parameter im Bereich von 2 bis 8 getestet. Die besten Ergebnisse bekamen wir bei einem COUNT von 4, was bedeutet, dass in jedem 4. Schritt ein Karten-Update durchgeführt wird.

%TODO bilder?

\subsubsection{Aktualisieren des Referenzscans (Janneke)}

Ein Problem mit unserer Implementation liegt darin, dass sich über die Zeit Fehler aufsummieren und die Scans immer weiter vom Original weg driften können. Ein möglicher Ansatz um dem entgegenzuwirken ist es den Referenzscan nicht in jedem Schritt, sondern nur ab und zu zu wechseln. Das heißt man arbeitet nicht jedes mal mit dem aktuellen und dem Scan aus dem vorherigen Schritt, sondern immer mit dem aktuellen und dem zuletzt festgelegten Referenzscan.

Also wird der Referenzscan, die ''Referenz-''Odometrie und sowohl der globale Rotationsoffset, als auch der globale Translationsoffset nur alle x-Schritte mit den aktuellen Werten überschrieben und nicht mehr am Ende jedes Schleifendurchgangs.

Dies bringt einen weiteren Update-Parameter der optimiert werden muss. Diese Optimierung erwies sich als relativ schwierig, da wir bereits einen COUNT-Parameter haben der bestimmt wie oft das Karten-Update überhaupt passiert.

\begin{figure}
	\centering
	\includegraphics[width=14cm]{refTest_c1_r2LINKS_c4_r1RECHTS_4min}
	\caption{Links: Karten-Update jeden Schritt, Referenzscan-Update alle 2 Schritte, 4 Minuten Fahrzeit; Rechts: Karten-Update alle 4 Schritte, Referenzscan-Update jeden Schritt, 4 Minuten Fahrzeit\newline}
	\label{fig:refTest}
\end{figure}

Bei der Optimierung des Update-Parameter haben wir verschiedene Count- und Update-Parameterkombinationen getestet. Dabei war auffällig, dass bei einem selteneren Referenzscan-Update das Verfahren schlechter funktioniert hat, wenn der Roboter engere Kurven gefahren ist, sich zum Beispiel in einer Sackgasse umgedreht hat. Dies ist dadurch zu erklären, dass der mehrere Scans alte Referenzscan und der aktuelle Scan wenig gleiche Teile des Raumes abdecken. Dadurch kann eine Korrelation der Beiden Scans nicht sinnvoll berechnet werden. Um dieses Problem ab zu mildern haben wir zunächst das Umdrehen des Roboters in einer Sackgasse blockierend um 180 Grad zu einem nicht blockierenden drehen auf der Stelle durch entgegengesetzte Radgeschwindigkeiten ersetzt. Außerdem haben wir insgesamt die Geschwindigkeit des Roboters reduziert, was sowohl der Version mit einem Referenzscan-Update in jedem Schritt als auch der mit weniger Updates geholfen hat.

Letztendlich ergaben unsere Test, dass unsere Implementation am besten funktioniert, wenn man entweder in jedem Schritt ein Karten-Update ausführt, aber den Referenzscan nur alle zwei Schritte updatet (siehe Abbildung~\ref{fig:refTest} links), oder nur jeden 4. Schritt ein Karten-Update ausführt, dafür aber in jedem dieser Schritte auch den Referenzscan wechselt (siehe Abbildung~\ref{fig:refTest} rechts). Da die zweite Version etwas besser zu funktionieren schien verwenden wir diese.


\subsection{Auflösung der Histogramme (Jan)}

Ein weiterer wichtiger Parameter ist die Auflösung der Histogramme.
Für das Winkelhistogramm natürlich wichtig, dass eine Rotation möglichst genau abgebildet werden kann, falls ein Maximum sich aber dann genau auf mehrere Bins aufteilt, kann es sein, dass das Maximum nicht gefunden wird. Hier ist es deshalb wichtig einen guten Mittelweg zu finden. Initial hatten wir für die Winkelhistogramme einen Binanzahl von 300, was einer Auflösung von 1,2 Grad pro Bin entspricht. Im weiteren Verlauf haben wir die Binanzahl auf 400 erhöht was eine leichte Verbesserung mit sich brachte. Eine Erhöhung der Binanzahl auf 500 hatte wiederum eine Verschlechterung zur Folge. Mit einer Anzahl von 550 Bins für das Winkelhistogramm haben wir letztendlich die besten Ergebnisse erzielen können. Dies entspricht einer Auflösung von etwa 0,65 Grad pro Bin.

Die Anzahl der Bins für die X- und Y-Histogramme muss zusätzlich auch immer noch auf die Distanzen in der Umgebung angepasst werden. Die optimalen Ergebnisse auf dem Roboter im Testbereich im Labor erzielten wir mit 500 Bins und einer maximalen Distanz von 6 Metern. Dies entspricht einer Auflösung von 1,2 cm pro Bin.

\subsection{Verrutschen der Karte}

Ab und zu kam es zu einem verrutschen der Karte. Wir nehmen an, dass hier durch eine Verzögerung des Netzwerks ein Scan zum falschen Zeitpunkt eingezeichnet wurde. Dieses Problem trat allerdings sehr selten auf und konnte auch an keinem weiteren Faktor festgemacht werden. Wie in Abbildung~\ref{fig:netzwerk} zu sehen ist, ist dadurch natürlich die Karte ungültig und es muss ein neuer Versuch gestartet werden.

\begin{figure}
	\centering
	\includegraphics[width=12cm]{netzwerk}
	\caption{Verrutschen der Karte, vermutlich verursacht durch eine Netzwerkverzögerung}
	\label{fig:netzwerk}
\end{figure}